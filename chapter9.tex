\section{Reversibilität}

\begin{karte}{Annahmen}
    Sei \((X_t)\) eine zeitstetige Markov-Kette,
    \begin{itemize}
        \item deren Intensitätsmatrix (Q1)-(Q3) erfüllt ist,
        \item die irreduzibel und pos. rekurrent mit stationärer 
        Verteilung \(\pi\) ist,
        \item wobei die Kette im Gleichgewicht, 
        d. h. \(\P(X_t = i) = \pi_i\) ist.
    \end{itemize} 
    Außerdem betrachten wir die ganze reelle Zeitachse.
\end{karte}

\begin{karte}{Stationär}
    Ein stochastischer Prozess 
    \((X_t)_{t\in \R}\) heißt \textit{stationär}, 
    falls 
    \[ (X_{t_1}, \ldots, X_{t_n}) \overset{d}{=}
    (X_{t_1+\Delta}, \ldots, X_{t_n + \Delta}) \]
    für alle \(-\infty < t_1 < \cdots < t_n < \infty, 
    \Delta \in \R, n\in\N\).
\end{karte}

\begin{karte}{Reversibilität Kriterium}

    Die Markov-Kette 
    \((X_t)\) ist reversibel \(\Leftrightarrow\) 
    die stationäre Verteilung erfüllt die 
    detailed balance equations, d. h. 
    \[ \pi_j q_{jk} = \pi_k q_{kj}. \]
\end{karte}

\begin{karte}{Gleichung mit \(A\subset S\)}
    Dann folgt für alle \(A\subset S\):
    \[ \sum_{j\in A} \sum_{k\in S\setminus A} 
    \pi_j q_{jk} = 
    \sum_{j\in A} \sum_{k\in S\setminus A} 
    \pi_k q_{kj}. \]
\end{karte}

\begin{karte}{Markov-Kette als Graph}
    Einer Markov-Kette 
    \((X_t) \) mit Zustandsraum \(S\) und 
    Intensitätsmatrix \(Q\) kann man einem Graphen zuordnen: 
    Ecken sind Zustände, Kante zwischen \(j\) und \(k\)
    (\(k\neq j\)) wenn \(q_{jk} > 0\) oder \(q_{kj} > 0\).

    Ist der assoziierte Graph ein Baum (zyklenfrei, 
    zusammenhängend), so ist die Markov-Kette reversibel.
\end{karte}