\section{Stationäre Verteilungen}

\begin{karte}{Maß}
    Eine Abbildung \( \abb{\nu}{S}{\R_{\geq 0}} \) nennen wir \textit{Naß}.
    Falls \( \sum_{i\in S} \nu(i) = 1 \), so nennen wir \(\nu\) eine Verteilung.
\end{karte}

\begin{karte}{Invariantes Maß, stationäre Verteilung}
    Sei \((X_n)\) eine Markov-Kette mit Übergangsmatrix \(P = (p_{ij})\). 
    Ein Maß \(\nu\) heißt \textit{invariant} für \(P\), falls 
    \(\nu \cdot P = \nu\), d. h. falls gilt: 
    \[ \sum_{i\in S} \nu(i) p_{ij} = \nu(j). \]
    Ist \(\nu\) eine Verteilung und invariant, so nennt man \(\nu\) 
    auch \textit{stationäre Verteilung} oder 
    \textit{Gleichgewichtsverteilung}.

    Ist \(S\) endlich, so kann jedes nichtnull invariante Maß normiert werden.

    Da \(\nu P^n = \nu\), gilt \(\P_\nu(X_n = j) = \nu(j)\).
\end{karte}

\begin{karte}{Detailed balance equations}
    Gelten die sogenannten \textit{detailed balance equations}, 
    d. h. gibt es eine Verteilung \(\pi\) auf \(S\) mit 
    \[ \pi(i) p_{ij} ) \pi(j) p_{ji}, \]
    dann ist \(\pi\) eine stationäre Verteilung.
\end{karte}

\begin{karte}{Eindeutigkeit invariantes Maß}
    Wir definieren für ein \(k\in S\) das Maß \(\gamma_k\) durch:
    \[ \gamma_k(i) := \E_k \left[ \sum_{n=1}^{T_k} 1_{[x_n = i]} \right], i \in S. \] 

    Sei \((X_n)\) irreduzibel und rekurrent, \(k\in S\). Dann gilt 
    \begin{enumerate}
        \item \(\gamma_k\) ist ein invariantes Maß,
        \item \(0 < \gamma_k < \infty\),
        \item \(\gamma_k\) ist das einzige Maß mit \(\gamma_k(k) = 1\) (eindeutig bis auf Vielfache).
    \end{enumerate}
    Ist \(S\) endlich und \(P\) irreduzibel, so existiert eine eindeutige Stationäre Verteilung.

    Ist \((X_n)\) irreduzibel und transient, so kann keine stationäre Verteilung existieren.
\end{karte}

\begin{karte}{Mittlere Rückkehrzeit}
    Für \(i\in S\) sei 
    \begin{align*}
        m_i &:= \E_i[T_i] = \sum_{n=1}^\infty n \cdot \P_i(T_i = n) + \infty \cdot (1-f_{ii}^*) \\
        &= \sum_{n=1}^\infty n \cdot f_{ii}^{(n)} + \infty (1-f_{ii}^*)
    \end{align*}
    die \textit{mittlere Rückkehrzeit} des Zustands \(i\).
\end{karte}

\begin{karte}{Positiv rekurrent, nullrekurrent}
    Ein Zustand \(i\in S\) heißt \textit{positiv rekurrent}, 
    falls \(m_i < \infty\), und \textit{nullrekurrent}, falls 
    \(i\) rekurrent und \(m_i = \infty\) ist.

    \begin{enumerate}
        \item Ist \(i\) positiv rekurrent, so ist \(\frac{1}{m_i}\) positiv, 
        ist \(i\) nullrekurrent, so ist \(\frac{1}{m_i} = 0\).
        \item Ist \(j\) transient, so folgt \(m_j = \infty\).
        \item Jeder positiv rekurrente Zustand ist auch rekurrent.
    \end{enumerate}
\end{karte}

\begin{karte}{Kriterium positive Rekurrenz}
    Sei \((X_n)\) eine irreduzible Markov-Kette. \\
    Die folgenden Aussagen sind äquivalent:
    \begin{enumerate}
        \item Es existiert eine stationäre Verteilung.
        \item Es gibt einen positiv rekurrenten Zustand \(i\in S\).
        \item Alle Zustände in \(S\) sind positiv rekurrent.
    \end{enumerate}

    Sind diese Bedingungen erfüllt, so ist die stationäre Verteilung eindeutig durch 
    \[ \pi(i) = \frac{1}{m_i}, i\in S \] 
    gegeben.
\end{karte}

\begin{karte}{Trichotomie für irreduzible Markov-Ketten}
    Eine irreduzible Markov-Kette gehört zu genau einem der folgenden Fälle:
    \begin{enumerate}
        \item Die Markov-Kette ist \textit{transient}, es gibt keine stationäre Verteilung.
        \item Die Markov-Kette ist \textit{nullrekurrent}. Für alle \(i,j\in S\) gilt:
        \[ \P_i(T_j < \infty) = 1 \text{ und } \E_i[T_j] = \infty \]
        und es gibt ein (bis auf Vielfache) eindeutiges invariantes Maß, aber keine stationäre Verteilung.
        \item Die Markov-Kette ist \textit{positiv rekurrent}. Für alle \(i,j\in S\) 
        ist \(\E_i[T_j] < \infty\) und es gibt eine stationäre Verteilung.
    \end{enumerate}

    Ist \(S\) endlich und die Markov-Kette irreduzibel, so ist sie automatisch positiv rekurrent.

    Ist \(\pi\) eine stationäre Verteilung, 
    so ist \(\pi(i)\) der durchschnittliche Bruchteil der Zeit, 
    den die Markov-Kette im Zustand \(i\) verbringt.
\end{karte}

\begin{karte}{Kriterium Transienz}
    Sei \((X_n)\) eine Markov-Kette auf \(S\). 
    Nehme einen beliebigen Zustand \(i_0 \in S\) und setze
    \(Q = (p_{ij}, i,j\in S \setminus\set{i_0})\).
    Die folgenden Aussagen sind äquivalent:
    \begin{enumerate}
        \item Die Markov-Kette ist transient.
        \item \(Qx = x, 0 \leq x_i \leq 1, i \in S\setminus\set{i_0}\) 
        besitzt eine Lösung \(\neq 0\).
    \end{enumerate}
\end{karte}