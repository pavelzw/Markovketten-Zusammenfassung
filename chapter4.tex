\section{Konvergenz stat. Verteilung}

\begin{karte}{Totalvariationsabstand}
    Seien \(\mu\) und \(\nu\) zwei Wahrscheinlichkeitsmaße auf \((S, \mathcal{P}(S))\).
    Der \textit{Totalvariationsabstand} \(d(\mu,\nu)\) zwischen \(\mu\) und \(\nu\) ist 
    \[ d(\mu,\nu) := \sup_{A\subset S} \abs{\mu(A) - \nu(A)}. \]

    Es gilt 
    \[ d(\mu,\nu) = \frac{1}{2} \sum_{i\in S} \abs{\mu(i) - \nu(i)}. \]
    Es folgt offensichtlich \(d(\mu,\nu) \leq 1\).

    Außerdem folgt 
    \[ \sum_{\substack{i\in S\\\mu(i)>\nu(i)}} (\mu(i) - \nu(i)) = d(\mu,\nu). \]
\end{karte}

\begin{karte}{Periode}
    Sei \((X_n)\) eine Markov-Kette mit Übergangsmatrix \(P\). Die 
    Periode \(d_i\) eines Zustandes ist gegeben durch 
    \[ d_i = ggT\set{n\in \N: p_{ii}^{(n)} > 0}. \]
    Ein Zustand \(i\) heißt \textit{aperiodisch}, wenn \(d_i = 1\).
\end{karte}

\begin{karte}{Eigenschaften periodische Markov-Kette}
    \(P\) ist irreduzibel und aperiodisch \(\Leftrightarrow\)
    für alle \(i,j\in S\) gibt es ein \(n_0\in \N\), sodass 
    für alle \(n\geq n_0\) gilt: \(p_{ij}^{(n)} > 0\).
\end{karte}

\begin{karte}{Konvergenzsatz}
    Sei \((X_n)\) eine irreduzible, aperiodisch und positiv rekurrente Markov-Kette mit Startverteilung \(\nu\) 
    und stationärer Verteilung \(\pi\). Dann gilt 
    \[ \limes{n} d(\nu P^n, \pi) = 0. \]

    Sind \(\mu, \nu\) verschiedene Startverteilungen, so gilt 
    \[ d(\nu P, \mu P) \leq d(\nu, \mu). \]
    Insbesondere ist \(d(\nu P, \pi) \leq d(\nu,\pi)\), d. h. der Totalvariationsabstand 
    von \(X_n\) zu \(\pi\) nimmt mit \(n\) ab.
\end{karte}

\begin{karte}{Mischungszeit}
    Sei \((X_n)\) eine irreduzible, aperiodische und positiv rekurrente 
    Markov-Kette mit stationärer Verteilung \(\pi\). Es sei 
    \[ d(n) := \sup_{i\in S} d(\delta_i P^n, \pi). \]
    Dann heißt \( t_{min}(\varepsilon) = \min \set{n\in \N: d(n)\leq \varepsilon} \) 
    für \(\varepsilon > 0\) \textit{Mischungszeit}.
\end{karte}

\begin{karte}{Kopplung}
    Seien \(\mu\) und \(\nu\) zwei Wahrscheinlichkeitsmaße auf \((S,\mathcal{P}(S))\). 
    Ein Zufallsvektor \((X,Y)\) auf \((S^2, \mathcal{P}(S^2))\) heißt \textit{Kopplung} 
    von \(\mu\) und \(\nu\), falls 
    \[ \P(X=i) = \mu(i), \P(Y=j)=\nu(j) \text{ für alle } i,j\in S, \]
    d. h. die Randverteilungen von \((X,Y)\) sind \(\mu\) bzw. \(\nu\).
\end{karte}

\begin{karte}{Totalvariationsabstand durch Kopplungen}
    Seien \(\mu\) und \(\nu\) zwei Wahrscheinlichkeitsmaße auf \((S,\mathcal{P}(S))\). 
    Dann gilt 
    \[ d(\mu,\nu) = \inf \set{\P(X\neq Y) : (X,Y) \text{ ist eine Kopplung von } \mu,\nu} \]
\end{karte}

\begin{karte}{Kopplung von Markov-Ketten}
    Eine Kopplung von Markov-Ketten \((X_n)\) und \((Y_n)\) mit Übergangsmatrix 
    \(P\) ist ein Prozess \((X_n,Y_n)\), sodass die Randprozesse \((X_n)\) und \((Y_n)\)
    Markov-Ketten mit Übergangsmatrix \(P\) sind und es gilt: 
    \[ X_m = Y_m \Rightarrow X_n = Y_n \text{ für alle } n\geq m. \]
\end{karte}

\begin{karte}{Eigenschaften Kopplung} % todo
    Sei \((X_n,Y_n)\) eine Kopplung von Markov-Ketten mit \(X_0 = i\), 
    \(Y_0 = j\). Sei 
    \[ \tau := \inf \set{n\in \N_0: X_n = Y_n}. \]
    Dann gilt 
    \[ d(\delta_i P^n, \delta_j P^n) \leq \P_{i,j}(\tau > n). \]
\end{karte}

\begin{karte}{\( \sup_{\sigma,\tau} \P(\tau > n) \)}
    Sei \(T := \inf \set{n\in \N: X_n = X_n'}\). Dann gilt unabhängig von den Startpermutationen \(\sigma\) 
    und \(\tau\)
    \[ \sup_{\sigma,\tau} \P(\tau > n) \leq \frac{\E_{\sigma,\tau} T}{n} 
    \leq \frac{\pi^2}{6} \frac{N^2}{n}. \]
\end{karte}

\begin{karte}{Konvergenzgeschwindigkeit}
    Sei \((X_n)\) eine irreduzible und aperiodische Markov-Kette auf 
    einem endlichen Zustandsraum \(S\) und \(\pi\) sei die stationäre Verteilung. 
    Dann gibt es ein \(\alpha \in (0,1)\) und ein \(C>0\), sodass
    \[ d(n) = \max_{i\in S} d(\delta_i P^n, \pi) \leq C \alpha^n \text{ für alle } n\in\N. \]

    Ist \(P\) Übergangsmatrix einer irreduziblen, aperiodischen Markov-Kette 
    mit endlichem Zustandsraum, dann ist \(1\) der größte Eigenwert 
    mit Eigenvektor \(e=(1,\ldots, 1)\) und 
    \[ P^n = \Pi + \mathcal{O}(n^{m_2 - 1} \abs{\lambda_2}^n), \]
    wobei \(\lambda_2\) der betragsmäßig zweitgrößte Eigenwert ist.
\end{karte}