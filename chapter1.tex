\section{Erste Begriffe}

\begin{karte}{Stochastischer Prozess}
    Eine Folge von Zufallsvariablen \((X_n)\) auf 
    \((\Omega, \mathcal{F}, \P)\) mit 
    \( \abb{X_n}{\Omega}{S} \) heißt \textit{stochastischer Prozess} 
    (in diskreter Zeit) mit Zustandsraum \(S\).
\end{karte}

\begin{karte}{Stochastische Matrix}
    Eine Matrix \(P = (p_{ij})\) heißt \textit{stochastische Matrix}, 
    falls \(p_{ij} \geq 0\) ist und \( \forall i \in S \) gilt 
    \( \sum_{j\in S} p_{ij} = 1 \).
\end{karte}

\begin{karte}{Markov-Kette}
    Sei \(P\) eine stochastische Matrix. 
    Eine Folge \(X_0, \ldots\) von \(S\)-wertigen 
    Zufallsvariablen heißt (homogene) \textit{Markov-Kette} 
    mit Übergangsmatrix \(P\), falls für alle \(n\in\N\) 
    und Zustände \(i_k\in S\) mit \( \P(X_0 = i_0, \ldots, X_n = i_n) > 0 \) 
    gilt 
    \[ \P(X_{n+1} = i_{n+1} \;|\; X_0 = i_0, \ldots, X_n = i_n) 
    = \P(X_{n+1} = i_{n+1} \;|\; X_n = i_n) = p_{i_n i_{n+1}}. \]

    Die \(p_{ij}\) heißen Übergangswahrscheinlichkeiten und die 
    Startverteilung \(\nu\) ist definiert durch \(\nu(i) = \P(X_0 = i)\) 
    für \(i\in S\).
\end{karte}

\begin{karte}{Eigenschaften von Markov-Ketten}
    Die folgenden Aussagen sind äquivalent:
    \begin{enumerate}
        \item \((X_n)\) ist eine Markov-Kette mit Übergangsmatrix \(P\).
        \item Für alle \(n\in \N_0, i_k \in S\) gilt 
        \[ \P(X_k = i_k, 0\leq k \leq n) = \P(X_0 = i_0) 
        \prod_{k=0}^{n-1} p_{i_ki_{k+1}}.  \]
        \item Für alle \(n\in \N_0, i_k\in S\) mit 
        \(\P(X_0 = i_0)>0\) gilt 
        \[ \P(X_k = i_k, 1\leq k \leq n \;|\; X_0 = i_0) = \prod_{k=0}^{n-1} p_{i_k i_{k+1}}. \]
        \item Für alle \(m,n\in\N_0, i_k \in S\) gilt 
        \[ \P(X_k = i_k, m \leq k \leq m+n) = \P(X_m = i_m) 
        \prod_{k=m}^{m+n-1} p_{i_k i_{k+1}}. \]
    \end{enumerate}
\end{karte}

\begin{karte}{Konstruktion}
    Seien \((Y_n)_n\) u. i. v. Zufallsvariablen mit Werten in einer beliebigen Menge \(Z\).
    Sei \(\abb{g}{S\times Z}{S}\) eine messbare Abbildung. 
    Definiere \( (X_n) \) mit 
    \[ X_0 = c \in S, \quad X_n = g(X_{n-1}, Y_n). \]
    Die so konstruierte Folge \((X_n)\) ist eine Markov-Kette mit Werten 
    in \(S\) und Übergangsmatrix \(P = (p_{ij})\) 
    mit \(p_{ij} = \P(g(i,Y_n) = j)\).
\end{karte}

\begin{karte}{\(n\)-Schritt-Übergangswahrscheinlichkeiten}
    Sei \(P\) eine stochastische Matrix. Dann heißen die Elemente 
    \(p_{ij}^{(n)}\) von \(P^n\) die \(n\)-Schritt-Übergangswahrscheinlichkeiten zu \(P\).
    Wir definieren \(P^0 := E\), also \(p_{ij}^{(0)} = \delta_{ij}\).
\end{karte}

\begin{karte}{\(n\)-Schritt-Übergangswahrscheinlichkeiten Eigenschaften}
    Sei \((X_n)\) eine Markov-Kette mit Übergangsmatrix \(P\). Dann gilt: 
    \begin{enumerate}
        \item \( \P(X_{n+m} = j \;|\; X_m = i) = p_{ij}^{(n)} \) für alle \(i,j\in S, m,n\in \N_0\) mit \(\P(X_m = i)>0\).
        \item \(P(X_n = j) = \sum_{i\in S} \P(X_0 = i) p_{ij}^{(n)}, j\in S, n\in \N\).
    \end{enumerate}
\end{karte}

\begin{karte}{Champan-Kolmogorov-Gleichung}
    \[ p_{ij}^{(n+m)} = \sum_{k\in S} p_{ik}^{(n)} p_{kj}^{(m)} \text{ für } i,j\in S. \]
\end{karte}