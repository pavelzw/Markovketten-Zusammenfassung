\section{Erste Begriffe}

\begin{karte}{Stochastischer Prozess}
    Eine Folge von Zufallsvariablen \((X_n)\) auf 
    \((\Omega, \mathcal{F}, \P)\) mit 
    \( \abb{X_n}{\Omega}{S} \) heißt \textit{stochastischer Prozess} 
    (in diskreter Zeit) mit Zustandsraum \(S\).
\end{karte}

\begin{karte}{Stochastische Matrix}
    Eine Matrix \(P = (p_{ij})\) heißt \textit{stochastische Matrix}, 
    falls \(p_{ij} \geq 0\) ist und \( \forall i \in S \) gilt 
    \( \sum_{j\in S} p_{ij} = 1 \).
\end{karte}

\begin{karte}{Markov-Kette}
    Sei \(P\) eine stochastische Matrix. 
    Eine Folge \(X_0, \ldots\) von \(S\)-wertigen 
    Zufallsvariablen heißt (homogene) \textit{Markov-Kette} 
    mit Übergangsmatrix \(P\), falls für alle \(n\in\N\) 
    und Zustände \(i_k\in S\) mit \( \P(X_0 = i_0, \ldots, X_n = i_n) > 0 \) 
    gilt 
    \[ \P(X_{n+1} = i_{n+1} \;|\; X_0 = i_0, \ldots, X_n = i_n) 
    = \P(X_{n+1} = i_{n+1} \;|\; X_n = i_n) = p_{i_n i_{n+1}}. \]

    Die \(p_{ij}\) heißen Übergangswahrscheinlichkeiten und die 
    Startverteilung \(\nu\) ist definiert durch \(\nu(i) = \P(X_0 = i)\) 
    für \(i\in S\).
\end{karte}

\begin{karte}{Eigenschaften von Markov-Ketten}
    Die folgenden Aussagen sind äquivalent:
    \begin{enumerate}
        \item \((X_n)\) ist eine Markov-Kette mit Übergangsmatrix \(P\).
        \item Für alle \(n\in \N_0, i_k \in S\) gilt 
        \[ \P(X_k = i_k, 0\leq k \leq n) = \P(X_0 = i_0) 
        \prod_{k=0}^{n-1} p_{i_ki_{k+1}}.  \]
        \item Für alle \(n\in \N_0, i_k\in S\) mit 
        \(\P(X_0 = i_0)>0\) gilt 
        \[ \P(X_k = i_k, 1\leq k \leq n \;|\; X_0 = i_0) = \prod_{k=0}^{n-1} p_{i_k i_{k+1}}. \]
        \item Für alle \(m,n\in\N_0, i_k \in S\) gilt 
        \[ \P(X_k = i_k, m \leq k \leq m+n) = \P(X_m = i_m) 
        \prod_{k=m}^{m+n-1} p_{i_k i_{k+1}}. \]
    \end{enumerate}
\end{karte}

\begin{karte}{Konstruktion}
    Seien \((Y_n)_n\) u. i. v. Zufallsvariablen mit Werten in einer beliebigen Menge \(Z\).
    Sei \(\abb{g}{S\times Z}{S}\) eine messbare Abbildung. 
    Definiere \( (X_n) \) mit 
    \[ X_0 = c \in S, \quad X_n = g(X_{n-1}, Y_n). \]
    Die so konstruierte Folge \((X_n)\) ist eine Markov-Kette mit Werten 
    in \(S\) und Übergangsmatrix \(P = (p_{ij})\) 
    mit \(p_{ij} = \P(g(i,Y_n) = j)\).
\end{karte}

\begin{karte}{\(n\)-Schritt-Übergangswahrscheinlichkeiten}
    Sei \(P\) eine stochastische Matrix. Dann heißen die Elemente 
    \(p_{ij}^{(n)}\) von \(P^n\) die \(n\)-Schritt-Übergangswahrscheinlichkeiten zu \(P\).
    Wir definieren \(P^0 := E\), also \(p_{ij}^{(0)} = \delta_{ij}\).
\end{karte}

\begin{karte}{\(n\)-Schritt-Übergangswahrscheinlichkeiten Eigenschaften}
    Sei \((X_n)\) eine Markov-Kette mit Übergangsmatrix \(P\). Dann gilt: 
    \begin{enumerate}
        \item \( \P(X_{n+m} = j \;|\; X_m = i) = p_{ij}^{(n)} \) für alle \(i,j\in S, m,n\in \N_0\) mit \(\P(X_m = i)>0\).
        \item \(P(X_n = j) = \sum_{i\in S} \P(X_0 = i) p_{ij}^{(n)}, j\in S, n\in \N\).
    \end{enumerate}
\end{karte}

\begin{karte}{Champan-Kolmogorov-Gleichung}
    \[ p_{ij}^{(n+m)} = \sum_{k\in S} p_{ik}^{(n)} p_{kj}^{(m)} \text{ für } i,j\in S. \]
\end{karte}

\subsection*{Zustände, Rekurrenz, Transienz}

\begin{karte}{Kommunizierend}
    Sei \((X_n)\) eine Markov-Kette mit Übergangsmatrix \(P = (p_{ij})\). 
    \begin{enumerate}
        \item Ein Zustand \(i\in S\) führt nach \(j\in S\) (kurz \(i \leadsto j\)), falls es ein \(n\in\N\) gibt mit \(p_{ij}^{(n)} > 0\).
        \item Ein Zustand \(i\in S\) kommuniziert mit \(j\in S\) (kurz \(i \leftrightarrow j\)), falls sowohl \(i \leadsto j\) als auch \(j \leadsto i\) gilt.
    \end{enumerate}

    Für \(i,j \in S\) sei \(i \sim j :\Leftrightarrow (i\leftrightarrow j) \vee (i = j)\). \\
    \(\sim \) ist eine Äquivalenzrelation auf \(S\).\\
    Die Äquivalenzklasse \(K(i) := \set{j\in S \;|\; i \sim j}\) enthält \(i\) und alle 
    mit \(i\) kommunizierenden Zustände.
\end{karte}

\begin{karte}{Abgeschlossenheit, Irreduzibilität}
    Sei \((X_n)\) eine Markov-Kette mit Übergangsmatrix \(P = (p_{ij})\).
    \begin{enumerate}
        \item \(J\subset S\) heißt \textit{abgeschlossen}, wenn es keine zwei Zustände 
        \(j\in J, i \in S \setminus J\) gibt mit \(j\leadsto i\).
        \item Die Markov-Kette \((X_n)\) bzw. die Übergangsmatrix \(P\) 
        heißen \textit{irreduzibel}, falls \(S\) nur aus einer Klasse besteht, 
        also für alle \(i,j\in S, i\neq j\), gilt \(i \leftrightarrow j\).
    \end{enumerate}

    \(J\subset S\) ist genau dann abgeschlossen, wenn \(p_{ij}, i,j\in J\) 
    stochastisch ist.
\end{karte}

\begin{karte}{Eintrittszeit}
    Sei 
    \[ T_i := \inf \set{n\in \N: X_n = i} \]
    die Ersteintrittszeit der Markov-Kette in den Zustand \(i\). \\
    Wir setzen dabei \(\inf \emptyset := \infty\).
\end{karte}

\begin{karte}{Eintrittswahrscheinlichkeiten}
    Sei für \(i,j\in S, n\in\N\): 
    \begin{align*}
        f_{ij}^{(n)} &:= \P(T_j = n \;|\; X_0 = i) = \P_i(T_j = n) \\
        &= \P(X_n = j, X_\nu \neq j \text{ für } 1 \leq \nu < n \;|\; X_0 = i) \\
        f_{ij}^{(0)} &:= 0.
    \end{align*}
    Offenbar ist \(f_{ij}^{(1)} = p_{ij}\).

    Wir definieren die Eintrittswahrscheinlichkeiten 
    \begin{align*}
        f_{ij}^* &:= \sum_{n=0}^\infty f_{ij}^{(n)} \\
        &= \sum_{n=0}^\infty \P_i(T_j = n) = \P_i(T_j < \infty) \\
        &= \P_i(\exists n\in \N: X_n = j) \in [0,1].
    \end{align*}
\end{karte}

\begin{karte}{Rekkurent, transient}
    Ein Zustand \(i\in S\) heißt \textit{rekurrent}, falls 
    \(f_{ii}^* = 1\) und \textit{transient} sonst.

    Ein Zustand \(i\in S\) ist rekurrent genau dann, wenn gilt 
    \[ \sum_{n=0}^\infty p_{ii}^{(n)} = \infty. \]
\end{karte}

\begin{karte}{Eintrittswahrscheinlichkeiten Gleichung}
    Für alle \(n\in \N, i,j\in S\) gilt: 
    \[ p_{ij}^{(n)} = \sum_{k=1}^n f_{ij}^{(k)} p_{jj}^{(n-k)}. \]
\end{karte}

\begin{karte}{Solidaritätsprinzip}
    Ist ein Zustand \(i\in S\) rekurrent (bzw. transient), so ist 
    jeder Zustand in \(K(i)\) rekurrent (bzw. transient).

    Ist \(i \in S\) rekkurent (bzw. transient), so sagen wir 
    \(K(i)\) ist rekurrent (bzw. transient). 
    Ist \((X_n)\) irreduzibel und ein \(i \in S\) rekkurent (bzw. transient), 
    so sagen wir \((X_n)\) ist rekurrent (bzw. transient).
\end{karte}