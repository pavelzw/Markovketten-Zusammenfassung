\section{Zustände, Rekurrenz, Transienz}

\subsection*{Klassifikation}

\begin{karte}{Kommunizierend}
    Sei \((X_n)\) eine Markov-Kette mit Übergangsmatrix \(P = (p_{ij})\). 
    \begin{enumerate}
        \item Ein Zustand \(i\in S\) führt nach \(j\in S\) (kurz \(i \leadsto j\)), falls es ein \(n\in\N\) gibt mit \(p_{ij}^{(n)} > 0\).
        \item Ein Zustand \(i\in S\) kommuniziert mit \(j\in S\) (kurz \(i \leftrightarrow j\)), falls sowohl \(i \leadsto j\) als auch \(j \leadsto i\) gilt.
    \end{enumerate}

    Für \(i,j \in S\) sei \(i \sim j :\Leftrightarrow (i\leftrightarrow j) \vee (i = j)\). \\
    \(\sim \) ist eine Äquivalenzrelation auf \(S\).\\
    Die Äquivalenzklasse \(K(i) := \set{j\in S \;|\; i \sim j}\) enthält \(i\) und alle 
    mit \(i\) kommunizierenden Zustände.
\end{karte}

\begin{karte}{Abgeschlossenheit, Irreduzibilität}
    Sei \((X_n)\) eine Markov-Kette mit Übergangsmatrix \(P = (p_{ij})\).
    \begin{enumerate}
        \item \(J\subset S\) heißt \textit{abgeschlossen}, wenn es keine zwei Zustände 
        \(j\in J, i \in S \setminus J\) gibt mit \(j\leadsto i\).
        \item Die Markov-Kette \((X_n)\) bzw. die Übergangsmatrix \(P\) 
        heißen \textit{irreduzibel}, falls \(S\) nur aus einer Klasse besteht, 
        also für alle \(i,j\in S, i\neq j\), gilt \(i \leftrightarrow j\).
    \end{enumerate}

    \(J\subset S\) ist genau dann abgeschlossen, wenn \(p_{ij}, i,j\in J\) 
    stochastisch ist.
\end{karte}

\begin{karte}{Eintrittszeit}
    Sei 
    \[ T_i := \inf \set{n\in \N: X_n = i} \]
    die Ersteintrittszeit der Markov-Kette in den Zustand \(i\). \\
    Wir setzen dabei \(\inf \emptyset := \infty\).
\end{karte}

\begin{karte}{Eintrittswahrscheinlichkeiten}
    Sei für \(i,j\in S, n\in\N\): 
    \begin{align*}
        f_{ij}^{(n)} &:= \P(T_j = n \;|\; X_0 = i) = \P_i(T_j = n) \\
        &= \P(X_n = j, X_\nu \neq j \text{ für } 1 \leq \nu < n \;|\; X_0 = i) \\
        f_{ij}^{(0)} &:= 0.
    \end{align*}
    Offenbar ist \(f_{ij}^{(1)} = p_{ij}\).

    Wir definieren die Eintrittswahrscheinlichkeiten 
    \begin{align*}
        f_{ij}^* &:= \sum_{n=0}^\infty f_{ij}^{(n)} \\
        &= \sum_{n=0}^\infty \P_i(T_j = n) = \P_i(T_j < \infty) \\
        &= \P_i(\exists n\in \N: X_n = j) \in [0,1].
    \end{align*}
\end{karte}

\begin{karte}{Rekkurent, transient}
    Ein Zustand \(i\in S\) heißt \textit{rekurrent}, falls 
    \(f_{ii}^* = 1\) und \textit{transient} sonst.

    Ein Zustand \(i\in S\) ist rekurrent genau dann, wenn gilt 
    \[ \sum_{n=0}^\infty p_{ii}^{(n)} = \infty. \]
\end{karte}

\begin{karte}{Eintrittswahrscheinlichkeiten Gleichung}
    Für alle \(n\in \N, i,j\in S\) gilt: 
    \[ p_{ij}^{(n)} = \sum_{k=1}^n f_{ij}^{(k)} p_{jj}^{(n-k)}. \]
\end{karte}

\begin{karte}{Solidaritätsprinzip}
    Ist ein Zustand \(i\in S\) rekurrent (bzw. transient), so ist 
    jeder Zustand in \(K(i)\) rekurrent (bzw. transient).

    Ist \(i \in S\) rekkurent (bzw. transient), so sagen wir 
    \(K(i)\) ist rekurrent (bzw. transient). 
    Ist \((X_n)\) irreduzibel und ein \(i \in S\) rekkurent (bzw. transient), 
    so sagen wir \((X_n)\) ist rekurrent (bzw. transient).

    Liegen \(i\) und \(j\) in derselben rekurrenten Klasse, 
    so gilt \( f_{ij}^* = f_{ji}^* = 1 \).
\end{karte}

\begin{karte}{Transienzkriterium}
    Für alle \(i,j \in S\) gilt: Wenn \(j\) transient ist, dann gilt 
    \[ \sum_{n=0}^\infty p_{ij}^{(n)} < \infty. \]
    Insbesondere ist \(\limes{n} p_{ij}^{(n)} = 0\).
\end{karte}

\begin{karte}{Eigenschaften Rekurrenzklassen}
    Ist eine Klasse \(K\subseteq S\) rekkurent, so ist \(K\) 
    abgeschlossen bzw. die Matrix \((p_{ij}, i,j\in K)\) ist stochastisch.
\end{karte}

\begin{karte}{Endliche Äquivalenzklassen}
    Ist die Klasse \(K\) endlich und abgeschlossen, so ist \(K\) rekurrent.

    Insbesondere gilt: Ist \(S\) endlich und \(P\) irreduzibel, so ist 
    die Markov-Kette rekurrent.
\end{karte}

\subsection*{Absorptionswahrscheinlichkeiten}

\begin{karte}{Austrittszeit}
    Zerlegen wir \(S\) in rekurrente Klassen \(K_1, \ldots, K_m\)
    und eine Menge von transienten Zuständen \(T\).

    Es sei \(\tau := \inf \set{n\geq 0 \;|\; X_n \notin T}\) die 
    Austrittszeit aus der Menge der transienten Zustände.

    Für \(i\in T, k \in T^C\) interessiert uns 
    \[ u_{ik} := \P_i(X_\tau = k). \]
\end{karte}

\begin{karte}{Austrittszeitwahrscheinlichkeit}
    Für \(i\in T, j \in T^C\) gilt:
    \[ u_{ij} = \sum_{k\in T} p_{ik} u_{kj} + p_{ij}. \]
\end{karte}

\begin{karte}{Fundamentalmatrix}
    Unterteilen wir \(P = (p_{ij})\) in 
    \[ P = \begin{pmatrix}
        Q & R \\ 0 & \tilde{P}
    \end{pmatrix}, \]
    wobei \(Q = (q_{ij}) = (p_{ij}, i,j\in T)\) und ist 
    \(U = (u_{ij})_{i\in T, j\in T^C}\),
    dann gilt \( U = (E - Q)^{-1} R \).

    Die sogenannte \textit{Fundamentalmatrix} \((E-Q)^{-1}\) 
    existiert.
\end{karte}

\subsection*{Absorptionsdauern}

\begin{karte}{Mittlere Absorptionsdauer}
    Sei \(\abb{g}{S}{\R}\). Wir definieren 
    \[ w_i := \E_i \left[ \sum_{n=0}^{\tau - 1} g(X_n) \right] \]

    Für \(i\in T, j\in T^C\) gilt: 
    \[ w_i = g(i) + \sum_{j\in T} p_{ij} w_j. \]

    Für \(g \equiv 1\) ergibt sich \(w_i = \E_i \tau\).
    
    Für \(g(i) = \delta_{ij}, j\in T\) ergibt sich \(w_i = \E_i \left[\sum_{n=0}^\infty 1_{\set{X_n = j}}\right]\).

    \(w\) lässt sich schreiben als 
    \[ w = (E - Q)^{-1} g. \]
\end{karte}

\begin{karte}{}

\end{karte}

\begin{karte}{}

\end{karte}

\begin{karte}{}

\end{karte}

\begin{karte}{}

\end{karte}

\begin{karte}{}

\end{karte}

\begin{karte}{}

\end{karte}